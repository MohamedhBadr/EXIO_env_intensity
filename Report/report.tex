\documentclass[a4paper,twoside]{article}
\usepackage{graphicx, fullpage, float, verbatim,amsmath, subcaption, minted}

\title{Evolution of CO2e multipliers over time: the case of French Agriculture}
\author{Molly Bazilchuk & Mohamed Badr}
\date{\today}


\begin{document}
\maketitle
\vspace{3cm}
\bibliographystyle{abbrv}

\section{Introduction}

Globally, food production is responsible for around one-quarter of the world's greenhouse gas emissions, while agriculture, forestry and land-use change account for around 18\%. Reducing agricultural emissions will therefore be key in limiting the extent of climate change. In terms of agricultural, France is the largest producer in Europe and is therefore an interest case for study when considering emissions intensity change over time. The French Ministry for Agriculture and Food stated in 2018 their goal to reduce emissions by 50\% by 2050 compared to 1990 levels. However, as is the case with the majority of government pledges, this deals with direct emissions rather than footprint or consumption-based emissions as is considered in input-output analysis. 

In this work, we perform an input-output analysis of the agricultural products of France. The multipliers in $kgCO_2e/Euro$ are considered in particular, in order to decouple changes in the production intensity from effects of the final demand. While many IO studies choose to aggregate products in input-output tables to more general categories, we have kept the original agricultural sectors to profit from the detail in the EXIOBASE data.

Few studies look at the evolution of multipliers over time. Liu et al. \cite{Liu2017} looked at the change of pollutants and carbon emissions multipliers in China in the period 2007 - 2012, considering overall changes in the different sectors of the economy  \cite{Liu2017} . Camanzi et al. 

Previous input-output studies considering the footprint of agriculture have done so on a more aggregated  and focus on the total footprint rather than the multipliers themselves \cite{Camanzi2017}. Emissions multipliers in China \cite{Liu2017}.

Will multipliers improve? Mechanism of improving efficiency/technology: genetic improvements, more efficient machinery, improved agro-chemicals and more efficient irrigation systems \cite{Schneider2009}. Same study shows increased efficiency of all inputs to agriculture but stabilizes around 1995 - have we reached a technical cap?

\section{Methods and data}

\subsection{Database}

The analysis was performed using EXIOBASE 3, the multi-regional input-output database with high sectoral resolution (200 products) and most accurate environmental extensions \cite{Stadler2018}. The scope of our model system was the 15 primary agricultural products. There are also secondary agricultural products (e.g. "dairy products" as opposed to "cattle"), but as our focus was on the production itself we excluded these. We chose to calculate multipliers for every other year from 1995 and onward to save computing power. 

For this research, we used the full EXIOBASE3 input-output dataset. The data can then be parsed and cleanly formatted into the relevant matrices using the Pymrio python package. EXIOBASE was chosen given that it is widely used and stands out with regards to sector resolution and environmental extensions. Due to computational limitations, we studied every other year starting from the year 1996 and ending at the year 2022.


\subsection{Multipliers in input-output}

For an input-output system, a multiplier is a quantity that, when multiplied with the final demand, gives the total emissions of a pollutant or other quantity such as energy consumption, labor use, etc. Multipliers $m$ are derived by performing the firm balance for some amount of emissions $F$, where $Z$ is the inter-industry flow matrix describing intermediate demand flowing between industries, and $x$ is the total output vector.

\begin{equation}
F + M Z = m \hat{x}
\end{equation}

We can then rearrange and express as a function of the Leontief inverse $L$ for a system.

\begin{equation}
M = F \hat{x}^{-1} (I - A)^{-1} = f \hat{x} L
\end{equation}


The $M$ matrix gives us an understanding of the impacts per dollar/euro, which in a broader sense displays the carbon intensity of given products/sectors. For a more elaborate explanation we point the reader to the EXIOBASE3 paper (Stadler et. al, 2018).

\subsection{Impacts}

We aggregate the impacts into GHG emissions “(GWP100) | Problem oriented approach: baseline (CML, 2001) | GWP100 (IPCC, 2007)”. This is done through characterization, which allows us to gain a broader perspective of all the GHG related impacts per product. After a primary analysis we then normalise all the multiplier values on the base year of 1996. This allows the research to better understand the evolution of multiplier values within a given time frame.


\subsection{Inflation}

To adjust for inflation, we use the Food and Agriculture Organisation of the United Nations (FAOSTAT) data on inflation in the French agricultural sector in the relevant time frame. Given the absence of product specific inflation statistics, this was the most relevant inflation data available. We normalise all prices on the base year 2015 as per FAOSTAT. Given that FAOSTAT only provides inflation figures from the year 2000 we used the inflation rate for the year 2000 for the years 1996 and 1998. More so, the inflation rate for the year 2022 has not been released yet (for the relevant sectors), we therefore use the inflation rate of the previous year (2021).

\section{Results}

Fig. \ref{fig:multipliers} shows the normalized evolution of the 14 agricultural multipliers from the year 1996 - 2021. Note that this data has not been adjusted for inflation, which would artificially increase the rate of reduction. The general trend is a decrease of the multipliers over time.

Calculate the rate of decrease??

In order to identify whether the 

\begin{figure}[H]
\centering
\includegraphics[width=0.5\textwidth]{output_line.png}
\caption{GHG emissions per year of French agricultural sectors.}\label{fig:GHG} 
\end{figure}

Can we fit the GHG emissions to a PAT framework? Might not be perfect since the agricultural products are being consumed by others than French citizens.

Compare to targets - but these are on direct emissions, can we consider only these as well?

\section{Discussion}

\bibliography{IOreport}

\appendix

\section{Python code}

\begin{minted}{python}

"""
Analysing he Carbon intensity of French Agricultural Products

Here we will just explore the data relevant to the Agricultural sector in France. 
and adjust to inflation to see its effect on multiplier value. 

"""

# %%
# Load Libraries
import pymrio
import pandas 
import numpy as np
import pandas as pd

# Define Inputs
Country = 'FR'
charact_table = pd.read_csv('public_char_factors (1).csv',  sep='\t')
ghg = 'GHG emissions (GWP100) | Problem oriented approach: baseline (CML, 2001) | GWP100 (IPCC, 2007)'


# Load Data

#Load EXIOBASE 1996
exio3_1996 = pymrio.parse_exiobase3(path='IOT_1996_pxp.zip')
exio3_1996.calc_all()
#Load EXIOBASE 1998
exio3_1998 = pymrio.parse_exiobase3(path='IOT_1998_pxp.zip')
exio3_1998.calc_all()
#Load EXIOBASE 2000
exio3_2000 = pymrio.parse_exiobase3(path='IOT_2000_pxp.zip')
exio3_2000.calc_all()
#Load EXIOBASE 2002
exio3_2002 = pymrio.parse_exiobase3(path='IOT_2002_pxp.zip')
exio3_2002.calc_all()
#Load EXIOBASE 2004
exio3_2004 = pymrio.parse_exiobase3(path='IOT_2004_pxp.zip')
exio3_2004.calc_all()
#Load EXIOBASE 2006
exio3_2006 = pymrio.parse_exiobase3(path='IOT_2006_pxp.zip')
exio3_2006.calc_all()
#Load EXIOBASE 2008
exio3_2008 = pymrio.parse_exiobase3(path='IOT_2008_pxp.zip')
exio3_2008.calc_all()
#Load EXIOBASE 2010
exio3_2010 = pymrio.parse_exiobase3(path='IOT_2010_pxp.zip')
exio3_2010.calc_all()
#Load EXIOBASE 2012
exio3_2012 = pymrio.parse_exiobase3(path='IOT_2012_pxp.zip')
exio3_2012.calc_all()
#Load EXIOBASE 2014
exio3_2014 = pymrio.parse_exiobase3(path='IOT_2014_pxp.zip')
exio3_2014.calc_all()
#Load EXIOBASE 2016
exio3_2016 = pymrio.parse_exiobase3(path='IOT_2016_pxp.zip')
exio3_2016.calc_all()
#Load EXIOBASE 2018
exio3_2018 = pymrio.parse_exiobase3(path='IOT_2018_pxp.zip')
exio3_2018.calc_all()
#load EXIOBASE 2020
exio3_2020 = pymrio.parse_exiobase3(path='IOT_2020_pxp.zip')
exio3_2020.calc_all()
#load EXIOBASE 2022
exio3_2022 = pymrio.parse_exiobase3(path='IOT_2022_pxp.zip')
exio3_2022.calc_all()


# %%
#Get the GHG equivilent multipliers

#Get GHG impacts in From French agriculture 1996
impacts_1996 = exio3_1996.satellite.characterize(charact_table, name="impacts")
ghg_1996 = impacts_1996.F.loc[[ghg]]
ghg_1996 = ghg_1996[Country]
ghg_1996 = ghg_1996.iloc[:, 0:15]
ghg_1996

#Get GHG impacts in From French agriculture 1998
impacts_1998 = exio3_1998.satellite.characterize(charact_table, name="impacts")
ghg_1998 = impacts_1998.F.loc[[ghg]]
ghg_1998 = ghg_1998[Country]
ghg_1998 = ghg_1998.iloc[:, 0:15]
ghg_1998

#Get GHG impacts in From French agriculture 2000
impacts_2000 = exio3_2000.satellite.characterize(charact_table, name="impacts")
ghg_2000 = impacts_2000.F.loc[[ghg]]
ghg_2000 = ghg_2000[Country]
ghg_2000 = ghg_2000.iloc[:, 0:15]
ghg_2000

#Get GHG impacts in From French agriculture 2002
impacts_2002 = exio3_2002.satellite.characterize(charact_table, name="impacts")
ghg_2002 = impacts_2002.F.loc[[ghg]]
ghg_2002 = ghg_2002[Country]
ghg_2002 = ghg_2002.iloc[:, 0:15]
ghg_2002

#Get GHG impacts in From French agriculture 2004
impacts_2004 = exio3_2004.satellite.characterize(charact_table, name="impacts")
ghg_2004 = impacts_2004.F.loc[[ghg]]
ghg_2004 = ghg_2004[Country]
ghg_2004 = ghg_2004.iloc[:, 0:15]
ghg_2004

#Get GHG impacts in From French agriculture 2006
impacts_2006 = exio3_2006.satellite.characterize(charact_table, name="impacts")
ghg_2006 = impacts_2006.F.loc[[ghg]]
ghg_2006 = ghg_2006[Country]
ghg_2006 = ghg_2006.iloc[:, 0:15]
ghg_2006

#Get GHG impacts in From French agriculture 2008
impacts_2008 = exio3_2008.satellite.characterize(charact_table, name="impacts")
ghg_2008 = impacts_2008.F.loc[[ghg]]
ghg_2008 = ghg_2008[Country]
ghg_2008 = ghg_2008.iloc[:, 0:15]
ghg_2008

#Get GHG impacts in From French agriculture 2010
impacts_2010 = exio3_2010.satellite.characterize(charact_table, name="impacts")
ghg_2010 = impacts_2010.F.loc[[ghg]]
ghg_2010 = ghg_2010[Country]
ghg_2010 = ghg_2010.iloc[:, 0:15]
ghg_2010

#Get GHG impacts in From French agriculture 2012
impacts_2012 = exio3_2012.satellite.characterize(charact_table, name="impacts")
ghg_2012 = impacts_2012.F.loc[[ghg]]
ghg_2012 = ghg_2012[Country]
ghg_2012 = ghg_2012.iloc[:, 0:15]
ghg_2012

#Get GHG impacts in From French agriculture 2014
impacts_2014 = exio3_2014.satellite.characterize(charact_table, name="impacts")
ghg_2014 = impacts_2014.F.loc[[ghg]]
ghg_2014 = ghg_2014[Country]
ghg_2014 = ghg_2014.iloc[:, 0:15]
ghg_2014

#Get GHG impacts in From French agriculture 2016
impacts_2016 = exio3_2016.satellite.characterize(charact_table, name="impacts")
ghg_2016 = impacts_2016.F.loc[[ghg]]
ghg_2016 = ghg_2016[Country]
ghg_2016 = ghg_2016.iloc[:, 0:15]
ghg_2016

#Get GHG impacts in From French agriculture 2018
impacts_2018 = exio3_2018.satellite.characterize(charact_table, name="impacts")
ghg_2018 = impacts_2018.F.loc[[ghg]]
ghg_2018 = ghg_2018[Country]
ghg_2018 = ghg_2018.iloc[:, 0:15]
ghg_2018

#Get GHG impacts in From French agriculture 2020
impacts_2020 = exio3_2020.satellite.characterize(charact_table, name="impacts")
ghg_2020 = impacts_2020.F.loc[[ghg]]
ghg_2020 = ghg_2020[Country]
ghg_2020 = ghg_2020.iloc[:, 0:15]


#Get GHG impacts in From French agriculture 2022
impacts_2022 = exio3_2022.satellite.characterize(charact_table, name="impacts")
ghg_2022 = impacts_2022.F.loc[[ghg]]
ghg_2022 = ghg_2022[Country]
ghg_2022 = ghg_2022.iloc[:, 0:15]

# %%
#Visualize

data = [ghg_1996,ghg_1998,ghg_2000,ghg_2002,ghg_2004,ghg_2006,ghg_2008, ghg_2010,ghg_2012,ghg_2014,ghg_2016,ghg_2018,ghg_2020,ghg_2022]

ghg_df = pd.concat(data).fillna(0)
ghg_df.index.values[[0,1, 2, 3, 4, 5, 6, 7, 8, 9, 10, 11,12,13]] = ['1996','1998', '2000','2002', '2004','2006','2008', '2010','2012','2014', '2016','2018','2020','2022']
ghg_df = ghg_df.T
ghg_df.plot(figsize = (10,10), kind='bar', title = 'GHG emissions')

#With Wool
ghg_df.plot(figsize=(15,15), kind='bar')

#without Wool
Trial_ghg_df = ghg_df.drop(index = 'Wool, silk-worm cocoons')
Trial_ghg_df.plot(figsize=(15,15), kind='bar',title='French Agriculture Multiplier Value - GHG')

#Line plot not including wool
ghg_t = Trial_ghg_df.T
ghg_t.plot.line(figsize=(15,15),title='French Agriculture Multiplier Value- GHG')


# %%
#get Total impacts from consumption based accounting

#1996
FR_D_cba_1996 = exio3_1996.impacts.D_cba[Country]
FR_D_cba_agr_1996 = FR_D_cba_1996.iloc[:, 0:15]
FR_D_cba_agr_ghg_1996 = FR_D_cba_agr_1996.loc[ghg]
FR_D_cba_agr_ghg_1996

#2000
FR_D_cba_2000 = exio3_2000.impacts.D_cba[Country]
FR_D_cba_agr_2000 = FR_D_cba_2000.iloc[:, 0:15]
FR_D_cba_agr_ghg_2000 = FR_D_cba_agr_2000.loc[ghg]
FR_D_cba_agr_ghg_2000

#2006
FR_D_cba_2006 = exio3_2006.impacts.D_cba[Country]
FR_D_cba_agr_2006 = FR_D_cba_2006.iloc[:, 0:15]
FR_D_cba_agr_ghg_2006 = FR_D_cba_agr_2006.loc[ghg]
FR_D_cba_agr_ghg_2006

#2010
FR_D_cba_2010 = exio3_2010.impacts.D_cba[Country]
FR_D_cba_agr_2010 = FR_D_cba_2010.iloc[:, 0:15]
FR_D_cba_agr_ghg_2010 = FR_D_cba_agr_2010.loc[ghg]
FR_D_cba_agr_ghg_2010

#2014
FR_D_cba_2014 = exio3_2014.impacts.D_cba[Country]
FR_D_cba_agr_2014 = FR_D_cba_2014.iloc[:, 0:15]
FR_D_cba_agr_ghg_2014 = FR_D_cba_agr_2014.loc[ghg]
FR_D_cba_agr_ghg_2014

#2018
FR_D_cba_2018 = exio3_2018.impacts.D_cba[Country]
FR_D_cba_agr_2018 = FR_D_cba_2018.iloc[:, 0:15]
FR_D_cba_agr_ghg_2018 = FR_D_cba_agr_2018.loc[ghg]
FR_D_cba_agr_ghg_2018

#2022
FR_D_cba_2022 = exio3_2022.impacts.D_cba[Country]
FR_D_cba_agr_2022 = FR_D_cba_2022.iloc[:, 0:15]
FR_D_cba_agr_ghg_2022 = FR_D_cba_agr_2022.loc[ghg]
FR_D_cba_agr_ghg_2022


# %%

#Bar Plot
D_cba_data = [FR_D_cba_agr_ghg_1996,FR_D_cba_agr_ghg_2000,FR_D_cba_agr_ghg_2006,FR_D_cba_agr_ghg_2010,FR_D_cba_agr_ghg_2014,FR_D_cba_agr_ghg_2018,FR_D_cba_agr_ghg_2022]

D_cba_data_t = pd.concat(D_cba_data, join='outer', axis=1).fillna(0)
D_cba_data_t = D_cba_data_t.reset_index()
D_cba_data_t.columns.values[[1, 2,3,4,5,6,7]] = ['1996','2000','2006','2010','2014','2018','2022']
D_cba_data_t = D_cba_data_t.set_index('sector')
D_cba_data_t.plot(figsize=(10,10),kind = 'bar',title='Consumption Based accounts')


"""
For more details and the data sources feel free to explore the github repository:
github.com/MohamedhBadr/EXIO_env_intensity 

The inflation script/data is also on the same github repository. 

"""
\end{minted}

\end{document}